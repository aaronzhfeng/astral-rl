\documentclass{article}

\usepackage[preprint]{neurips_2025}

\usepackage[utf8]{inputenc}
\usepackage[T1]{fontenc}
\usepackage{hyperref}
\usepackage{url}
\usepackage{booktabs}
\usepackage{amsfonts}
\usepackage{amsmath}
\usepackage{amssymb}
\usepackage{nicefrac}
\usepackage{microtype}
\usepackage{xcolor}
\usepackage{graphicx}
\usepackage{algorithm}
\usepackage{algorithmic}
\usepackage{subcaption}
\usepackage{multirow}
\graphicspath{{./asset/figure/}}

% Custom commands
\newcommand{\astral}{\textsc{Astral}}
\newcommand{\method}{\astral}
\newcommand{\R}{\mathbb{R}}
\newcommand{\E}{\mathbb{E}}
\newcommand{\bh}{\mathbf{h}}
\newcommand{\bz}{\mathbf{z}}
\newcommand{\bw}{\mathbf{w}}
\newcommand{\bA}{\mathbf{A}}

\title{ASTRAL: Stable Test-Time Adaptation via Abstraction-Structured Gating for Non-Stationary Reinforcement Learning}

\author{%
  Anonymous Author(s) \\
  Institution \\
  \texttt{anonymous@institution.edu}
}

\begin{document}

\maketitle

\begin{abstract}
Reinforcement learning agents deployed in non-stationary environments must adapt to changing dynamics without catastrophic forgetting of previously learned behaviors. We introduce \method{} (\textbf{A}bstraction-\textbf{S}tructured \textbf{T}est-time \textbf{R}einforcement \textbf{A}daptation \textbf{L}ayer), an architecture that maintains a bank of learnable abstraction vectors combined through a lightweight gating mechanism. At test time, only the gating network ($\sim$4,300 parameters) is adapted, while the policy remains frozen. Through systematic comparison against full fine-tuning on a non-stationary CartPole environment, we demonstrate that \method{}'s gating-only adaptation achieves \textbf{10$\times$ less catastrophic forgetting} compared to full parameter adaptation. While full fine-tuning achieves higher peak improvement on individual modes (+77 vs +11), it causes severe performance collapse on non-adapted modes ($-143$ vs $-0.2$). We further show that gating adaptation maintains consistent, low-variance performance across all episode budgets (1-50), whereas policy-head and full fine-tuning exhibit high variance with frequent catastrophic failures. Our results suggest that \method{} is particularly suited for \emph{risk-averse} deployment scenarios where stable adaptation across multiple modes is more important than maximum single-mode improvement.
\end{abstract}

%==============================================================================
\section{Introduction}
\label{sec:intro}
%==============================================================================

Real-world reinforcement learning applications must contend with non-stationary environments where dynamics change over time \citep{padakandla2020survey}. A robot trained in simulation encounters different friction coefficients in deployment. An autonomous vehicle must adapt to varying weather conditions. Traditional approaches to this challenge include domain randomization \citep{tobin2017domain}, meta-learning \citep{finn2017maml}, and context-conditioned policies \citep{hallak2015contextual}.

A critical but often overlooked consideration is \emph{how} adaptation affects performance on other environment modes. An agent that perfectly adapts to Mode A but forgets Mode B may be worse than an agent that moderately handles both. This is especially important in multi-modal deployments where the agent must maintain competence across all conditions.

We introduce \method{}, an architecture designed for \textbf{stable, forgetting-resistant adaptation}. The key insight is to decompose the agent's representation into a bank of abstraction slots combined through a lightweight gating network. At test time, only the gating weights are adapted:

\begin{itemize}
    \item \textbf{Stability}: With the policy frozen, adaptation cannot catastrophically destroy learned behaviors
    \item \textbf{Efficiency}: Only $\sim$4,300 parameters (8\% of total) are updated
    \item \textbf{Predictability}: Low-variance adaptation across all episode budgets
\end{itemize}

Our contributions are:
\begin{enumerate}
    \item The \method{} architecture combining abstraction banks with FiLM modulation (Section~\ref{sec:method})
    \item Discovery and mitigation of \emph{slot collapse} via slot dropout (Section~\ref{sec:collapse})
    \item \textbf{Fair comparison experiments} demonstrating that gating adaptation provides 10$\times$ less forgetting than full fine-tuning (Section~\ref{sec:fair_comparison})
    \item Analysis showing gating is the \emph{safest} adaptation strategy across varying episode budgets (Section~\ref{sec:fewshot})
\end{enumerate}

%==============================================================================
\section{Related Work}
\label{sec:related}
%==============================================================================

\paragraph{Non-Stationary RL.}
Hidden-mode MDPs \citep{choi2000hidden} and context-conditioned policies \citep{hallak2015contextual} address non-stationarity by conditioning on inferred context. Our work focuses on \emph{test-time} adaptation rather than inference of hidden modes.

\paragraph{Continual Learning and Catastrophic Forgetting.}
Catastrophic forgetting \citep{mccloskey1989catastrophic, kirkpatrick2017overcoming} occurs when neural networks lose previously learned knowledge when trained on new tasks. Elastic Weight Consolidation (EWC) \citep{kirkpatrick2017overcoming} and Progressive Neural Networks \citep{rusu2016progressive} address this in supervised learning. Our gating-only adaptation provides a structural solution: the policy cannot forget because it is never updated.

\paragraph{Meta-Learning.}
MAML \citep{finn2017maml} learns initializations for fast fine-tuning. RL$^2$ \citep{duan2016rl2} uses recurrent networks for in-context adaptation. Our approach differs by \emph{freezing} most parameters rather than fine-tuning all of them.

\paragraph{Mixture of Experts.}
MoE architectures \citep{jacobs1991adaptive, shazeer2017outrageously} route inputs to specialized sub-networks. \method{} uses soft attention over shared abstraction vectors rather than discrete routing, and modulates via FiLM \citep{perez2018film} rather than output combination.

%==============================================================================
\section{Method}
\label{sec:method}
%==============================================================================

\subsection{Architecture Overview}

\method{} consists of four components:
\begin{enumerate}
    \item \textbf{Input Projection}: Maps observation $o_t$, previous action $a_{t-1}$, and reward $r_{t-1}$ to embedding
    \item \textbf{Recurrent Backbone}: GRU maintaining hidden state $\bh_t$
    \item \textbf{Abstraction Bank}: Produces abstraction $\bz_t$ and weights $\bw_t$ from $\bh_t$
    \item \textbf{FiLM-Modulated Heads}: Policy and value heads modulated by abstraction
\end{enumerate}

\subsection{Abstraction Bank}

The abstraction bank maintains $K$ learnable vectors $\bA = [\mathbf{a}_1, \ldots, \mathbf{a}_K] \in \R^{K \times d}$. Given context $\bh_t$, the gating network computes:
\begin{align}
    \bw_t &= \text{softmax}(W_g \cdot \text{ReLU}(W_h \bh_t) / \tau) \\
    \bz_t &= \sum_{k=1}^K w_{t,k} \mathbf{a}_k
\end{align}

\subsection{FiLM Modulation}

Following \citet{perez2018film}:
\begin{align}
    \gamma, \beta &= \text{MLP}(\bz_t) \\
    \bh'_t &= \gamma \odot \bh_t + \beta
\end{align}

\subsection{Training with Slot Dropout}

We train using PPO \citep{schulman2017proximal} with \textbf{slot dropout}: during training, each slot weight is zeroed with probability $p$ and weights are renormalized. This prevents any single slot from dominating and creates functionally diverse abstractions.

\subsection{Test-Time Adaptation}

At test time, we freeze all parameters except the gating network ($W_g$, $W_h$) and adapt using REINFORCE:
\begin{align}
    \nabla_\theta J(\theta) \approx \sum_t \nabla_\theta \log \pi_\theta(a_t|s_t) \cdot G_t
\end{align}
This adapts only $\sim$4,300 parameters while preserving the learned policy.

%==============================================================================
\section{Experimental Setup}
\label{sec:setup}
%==============================================================================

\subsection{Environment: Non-Stationary CartPole}

We modify CartPole \citep{brockman2016gym} with three physics modes:

\begin{table}[h]
\centering
\caption{Non-Stationary CartPole Environment Modes}
\label{tab:env_modes}
\begin{tabular}{@{}lccc@{}}
\toprule
\textbf{Mode} & \textbf{Gravity} & \textbf{Pole Length} & \textbf{Pole Mass} \\
\midrule
0 (Standard) & 9.8 & 0.5 & 0.1 \\
1 (Long Pole) & 10.8 & 0.6 & 0.1 \\
2 (Heavy Pole) & 9.8 & 0.4 & 0.2 \\
\bottomrule
\end{tabular}
\end{table}



\subsection{Baselines}

\begin{itemize}
    \item \textbf{SB3 PPO}: Stable-Baselines3 PPO \citep{stable-baselines3} trained on the same environment (100k steps, ~443 mean return)
    \item \textbf{Full Fine-Tuning}: Adapting all ASTRAL parameters at test time
    \item \textbf{Policy-Head Fine-Tuning}: Adapting only the policy head (~4,300 params, matched to gating)
\end{itemize}

\subsection{Adaptation Protocol}

For each adaptation experiment:
\begin{enumerate}
    \item Evaluate on target mode (20 episodes) $\rightarrow$ ``Before'' score
    \item Adapt for $N$ episodes on target mode
    \item Evaluate again (20 episodes) $\rightarrow$ ``After'' score
    \item For forgetting tests: also evaluate on non-adapted modes
\end{enumerate}

%==============================================================================
\section{Slot Collapse and Mitigation}
\label{sec:collapse}
%==============================================================================

\subsection{The Slot Collapse Problem}

Without regularization, \method{} converges to using a single abstraction slot regardless of environment mode:

\begin{table}[h]
\centering
\caption{Slot Collapse: Weight distributions without regularization}
\label{tab:collapse}
\begin{tabular}{@{}lcccc@{}}
\toprule
\textbf{Model} & \textbf{Slot 0} & \textbf{Slot 1} & \textbf{Slot 2} & \textbf{Collapsed?} \\
\midrule
Default (seed 42) & 0.00 & \textbf{1.00} & 0.00 & Yes \\
Default (seed 123) & 0.00 & \textbf{1.00} & 0.00 & Yes \\
Default (seed 456) & 0.00 & 0.00 & \textbf{1.00} & Yes \\
\midrule
With Regularization & 0.56 & 0.41 & 0.02 & Partial \\
\bottomrule
\end{tabular}
\end{table}



\subsection{Slot Dropout as Solution}

We found that \textbf{slot dropout} during training (probability $p=0.3$) creates diverse, functionally useful abstractions:

\begin{table}[h]
\centering
\caption{Effect of Slot Dropout on TTA Performance}
\label{tab:slot_dropout}
\begin{tabular}{@{}lccc@{}}
\toprule
\textbf{Configuration} & \textbf{Base Return} & \textbf{TTA Improvement} & \textbf{Status} \\
\midrule
Slot Dropout (p=0.3) & 187.4 & \textbf{+11.4} & $\checkmark$ Best \\
Slot Dropout (p=0.5) & 258.8 & -4.7 & Too aggressive \\
Strong Regularization & 490.5 & -3.0 & Ceiling effect \\
Diverse Strong & 314.0 & -64.6 & Catastrophic \\
Collapsed Default & 499.5 & -4.8 & No diversity \\
\bottomrule
\end{tabular}
\end{table}



Key insight: Regularization (contrastive loss, load balancing) creates \emph{numerical} diversity but not \emph{functional} diversity. Slot dropout forces the network to learn useful representations in all slots.

%==============================================================================
\section{Fair Comparison Experiments}
\label{sec:fair_comparison}
%==============================================================================

We designed four experiments to fairly compare \method{}'s gating adaptation against alternatives.

\subsection{Experiment A: Parameter-Matched Comparison}

\textbf{Question}: Is gating adaptation better than policy-head adaptation with the same parameter budget (~4,300 params)?

\begin{table}[h]
\centering
\caption{Experiment A: Parameter-Matched Comparison ($\sim$4,300 params each)}
\label{tab:exp_a}
\begin{tabular}{@{}lcccc@{}}
\toprule
\textbf{Method} & \textbf{Mode 0} & \textbf{Mode 1} & \textbf{Mode 2} & \textbf{Average} \\
\midrule
\textbf{Gating} & -2.7 & -7.9 & \textbf{+14.7} & +1.3 \\
Policy Head & -2.4 & \textcolor{red}{-105.0} & +10.0 & -32.5 \\
\bottomrule
\end{tabular}
\vspace{0.5em}
\small{Gating avoids catastrophic drops; policy-head causes -105 on Mode 1.}
\end{table}



\textbf{Finding}: Gating is more stable. Policy-head adaptation caused -105 drop on Mode 1 (catastrophic forgetting), while gating's worst case was -7.9.

\subsection{Experiment B: Catastrophic Forgetting Test}

\textbf{Question}: How much do non-adapted modes degrade when adapting to one mode?

\textbf{Protocol}: Adapt to Mode 0 only, then evaluate on all modes.

\begin{table}[h]
\centering
\caption{Experiment B: Catastrophic Forgetting (adapt Mode 0, evaluate all)}
\label{tab:exp_b}
\begin{tabular}{@{}lcccc@{}}
\toprule
\textbf{Method} & \textbf{Mode 0 After} & \textbf{Mode 1 $\Delta$} & \textbf{Mode 2 $\Delta$} & \textbf{Total Forgetting} \\
\midrule
\textbf{Gating} & 163.9 & \textbf{-0.2} & -25.6 & \textbf{-25.8} \\
Full Fine-Tune & 62.9 & \textcolor{red}{-143.5} & \textcolor{red}{-106.6} & \textcolor{red}{-250.1} \\
\bottomrule
\end{tabular}
\vspace{0.5em}
\small{\textbf{Key Result}: Gating causes \textbf{10$\times$ less forgetting} than full fine-tuning.}
\end{table}



\textbf{Finding}: \textbf{Gating causes 10$\times$ less forgetting} (-25.8 total) compared to full fine-tuning (-250.1 total). Mode 1 is almost perfectly preserved with gating (-0.2 vs -143.5).

\subsection{Experiment C: Few-Shot Adaptation Speed}

\textbf{Question}: Which method adapts fastest with limited episodes?

\begin{table}[h]
\centering
\caption{Experiment C: Few-Shot Adaptation Speed (improvement vs baseline)}
\label{tab:exp_c}
\begin{tabular}{@{}lccc@{}}
\toprule
\textbf{Episodes} & \textbf{Gating} & \textbf{Policy Head} & \textbf{Full} \\
\midrule
1 & +12.1 & +3.5 & \textcolor{red}{-50.4} \\
3 & +11.7 & +50.6 & +21.3 \\
5 & +4.2 & \textbf{+155.7} & -10.4 \\
10 & \textbf{+15.2} & -19.0 & -4.0 \\
20 & +9.0 & -58.1 & \textcolor{red}{-78.0} \\
30 & -6.3 & \textcolor{red}{-122.7} & +1.1 \\
\midrule
\textbf{Variance} & Low & \textbf{Very High} & High \\
\textbf{Worst Case} & -6.3 & -122.7 & -78.0 \\
\bottomrule
\end{tabular}
\end{table}



\begin{figure}[h]
\centering
\includegraphics[width=0.9\textwidth]{fewshot_comparison.png}
\caption{\textbf{Few-Shot Adaptation Comparison}: Gating (blue) shows consistent, low-variance improvement across all episode budgets. Policy-head (orange) peaks at 5 episodes then catastrophically collapses. Full fine-tuning (green) is highly unstable with frequent failures.}
\label{fig:fewshot}
\end{figure}



\textbf{Finding}: 
\begin{itemize}
    \item \textbf{Gating}: Consistent positive improvement (1-20 episodes), never catastrophic
    \item \textbf{Policy-head}: Peaks at 5 episodes (+155.7) but collapses after 20 (-122.7)
    \item \textbf{Full}: Highly unstable, frequent catastrophic failures
\end{itemize}

\subsection{Experiment D: Extreme Mode Differences}

\textbf{Question}: How do methods perform when modes are more different (gravity 5-25, length 0.3-0.8)?

\begin{table}[h]
\centering
\caption{Experiment D: Extreme Mode Differences (gravity 5-25, length 0.3-0.8)}
\label{tab:exp_d}
\begin{tabular}{@{}lcccc@{}}
\toprule
\textbf{Method} & \textbf{Mode 0 $\Delta$} & \textbf{Mode 1 $\Delta$} & \textbf{Mode 2 $\Delta$} & \textbf{Average} \\
\midrule
\textbf{Gating} & +27.6 & -0.4 & -1.8 & \textbf{+8.5} \\
Full Fine-Tune & +124.9 & \textcolor{red}{-103.0} & -27.6 & -1.8 \\
\bottomrule
\end{tabular}
\vspace{0.5em}
\small{Full achieves higher Mode 0 gain but catastrophically forgets Mode 1.}
\end{table}



\textbf{Finding}: On extreme modes, gating maintains positive average improvement (+8.5) while full fine-tuning degrades (-1.8 average, -103 on Mode 1).

%==============================================================================
\section{Analysis}
\label{sec:analysis}
%==============================================================================

\subsection{Why Gating Provides Stability}

The frozen policy acts as an ``anchor'' that cannot be corrupted. The gating network can only \emph{reweight} existing abstractions, not create new behaviors. This structural constraint limits both upside (maximum improvement) and downside (catastrophic failure).

\subsection{When to Use \method{}}

\begin{table}[h]
\centering
\caption{When to Use Each Adaptation Strategy}
\label{tab:when_to_use}
\begin{tabular}{@{}lll@{}}
\toprule
\textbf{Scenario} & \textbf{Best Method} & \textbf{Reason} \\
\midrule
Multi-mode deployment & \textbf{Gating} & Prevents forgetting \\
Unknown episode budget & \textbf{Gating} & Consistent performance \\
Risk-averse setting & \textbf{Gating} & No catastrophic failures \\
\midrule
Single-mode optimization & Full Fine-Tune & Maximum improvement \\
Exactly 5 episodes & Policy Head & Peak performance \\
\bottomrule
\end{tabular}
\end{table}



\subsection{Limitations}

\begin{enumerate}
    \item \textbf{Lower peak improvement}: Full fine-tuning achieves +77 vs gating's +11
    \item \textbf{Requires slot dropout}: Without it, slot collapse prevents meaningful TTA
    \item \textbf{Single environment}: Results may not generalize beyond CartPole
\end{enumerate}

%==============================================================================
\section{Conclusion}
\label{sec:conclusion}
%==============================================================================

We introduced \method{}, an architecture for stable test-time adaptation in non-stationary RL. Our key finding is that \textbf{stability and maximum improvement are fundamentally in tension}. Full fine-tuning achieves higher peak gains but causes 10$\times$ more forgetting and frequent catastrophic failures.

\method{}'s gating-only adaptation provides:
\begin{itemize}
    \item \textbf{10$\times$ less forgetting} on non-adapted modes
    \item \textbf{Consistent performance} across all episode budgets
    \item \textbf{No catastrophic failures} in any tested configuration
\end{itemize}

We recommend \method{} for \emph{risk-averse} deployments where maintaining competence across all modes is more important than maximizing single-mode performance. Future work should explore functional diversity mechanisms beyond slot dropout and extend to more complex environments.

%==============================================================================
\bibliographystyle{plainnat}
\bibliography{references}

%==============================================================================
\appendix
\section{Additional Results}
%==============================================================================

\subsection{Training Comparison}

\begin{figure}[h]
\centering
\includegraphics[width=0.8\textwidth]{training_comparison.png}
\caption{\textbf{Training Performance}: ASTRAL successfully learns the task (~490 return) while the original GRU baseline fails entirely (~24 return). SB3 PPO baseline achieves ~443 return.}
\label{fig:training}
\end{figure}



\subsection{Slot Weight Distributions}

\begin{figure}[h]
\centering
\includegraphics[width=0.9\textwidth]{weight_distributions.png}
\caption{\textbf{Slot Weight Distributions}: Without regularization (left), weights collapse to a single slot. With slot dropout (right), weights are distributed across multiple slots, enabling meaningful test-time adaptation.}
\label{fig:weight_dist}
\end{figure}



\subsection{Extreme Few-Shot Results}

\begin{table}[h]
\centering
\caption{Extreme Few-Shot: Adaptation on Extreme Modes (Episodes 1-50)}
\label{tab:extreme_fewshot}
\small
\begin{tabular}{@{}l|ccc|ccc|ccc@{}}
\toprule
& \multicolumn{3}{c|}{\textbf{Mode 0 (Easy)}} & \multicolumn{3}{c|}{\textbf{Mode 1 (Medium)}} & \multicolumn{3}{c}{\textbf{Mode 2 (Hard)}} \\
\textbf{Budget} & Gating & P.Head & Full & Gating & P.Head & Full & Gating & P.Head & Full \\
\midrule
1 & +24.5 & +109.4 & +156.3 & +17.1 & +52.1 & +25.0 & +15.4 & +11.6 & +2.0 \\
5 & +28.1 & +193.5 & -181.0 & -24.7 & -12.9 & +98.4 & +13.7 & -10.1 & +26.3 \\
10 & -3.5 & +140.1 & -3.5 & -23.7 & -96.5 & +96.2 & +9.7 & +21.1 & -37.4 \\
30 & +12.3 & -181.8 & +108.5 & -10.5 & -91.4 & -38.6 & +17.2 & +29.6 & +27.1 \\
50 & +15.4 & -51.9 & -127.1 & -16.2 & -28.7 & +30.3 & +0.3 & +21.5 & -42.8 \\
\bottomrule
\end{tabular}
\vspace{0.5em}
\small{Gating shows lowest variance across all conditions. Policy-head peaks at 5 episodes then collapses.}
\end{table}



\end{document}

